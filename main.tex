\documentclass{article}
\usepackage[utf8]{inputenc}

\title{reverse}
\author{wright.aprilm }
\date{October 2019}

\begin{document}

\maketitle
\section{Introduction}

Estimating phylogenetic trees has emerged as one of the predominant challenges in comparative biology and paleontology.
Phylogenetic trees provide us with the historical context in which traits and organisms evolved.
There is abundant evidence that trying to understand trait evolution without a phylogenetic tree is deeply misleading (cite a bunch of stuff here).
In particular, paleontological data are crucially important in comparative analyses (cite Slater and others).
As such, inclusion of paleontological data is being used in a wider variety of types of studies, and across more disciplines.

Despite the important role fossils play in phylogenetic inference, adoption of model-based method (such as maximum likelihood and Bayesian analyses) has been sluggish in the geosciences.
For many years, the only published Bayesian model was the Mk model (Lewis), an extremely simple model of evolution based on the Jukes-Cantor model of sequence evolution.
Methods became available to relax the models' restrictive assumptions through the use of Bayesian priors.
At the same time, methods for working with continuous trait data, and for incorporating fossils more completely in divergence time estimation have been published.

Many of these new methods are quite complex, modeling aspects of a project that may be hierarchical, and in which submodels may have complex dependencies on one another.
Historically, users of a phylogenetics software have been able to choose from a set of  models that the developers implemented.
Extending a model may or may not be possible, depending on how the software was coded. 
Additionally, many software packages have default values on parameters or priors in a model, which reduces the transparency of an analysis.
This type of project organization, in which the researcher is reliant on the computational scientists to do their work is inefficient.

The phylogenetics software RevBayes represents an attempt to reconfigure the way phylogenetics software is written.
RevBayes implements a statistical computing language, similar to R, called Rev.
Rev implements a large variety of probability distributions, as well as mathematical operations, such as Markov Chain Monte Carlo analysis.
Using Rev, an infinite combinations of models, priors and data can be assembled into analysis workflows.
A researcher who has a new idea to analyse their data, then, does not have to wait for a developer to implement their idea.
Likewise, assembling a model from all of its constituent pieces means there are no defaults, enabling a radical transparency in phylogenetic analysis. 

At the same time, asking researchers and students to learn a new programming language means asking them to take on a large cognitive load.
RevBayes can be used in a variety of ways: an RStudio interface, Jupyter Notebook, or the command line.
This manuscript will discuss techniques for teaching with RevBayes in a classroom or workshop context, with special attention paid to when and how to make use of the different interfaces.




\section{The Context of Systematics Education}

\begin{itemize}
    \item i.e., why do we even need a GUI
    \item Some biology disciplines routinely use server and command-line interfaces
    \item This is less common in others
    \item In courses where the content, not the code, is the focus, GUIs more common.
    \item Not uncommon to have courses with multiple skill levels among learners
    \item RevBayes relatively high cognitive load
    \item Python is more common in large data \& bioinformatics
    \item R more common in paleo \& comparative biology
\end{itemize}

\section{RevStudio}

How does this work?  
How can this tool be used in making websites, manuals, etc == interoperability

\section{RevNotebook}

How does this work? 

\section{A Comparison of the three interfaces}

When I use which?


\end{document}
