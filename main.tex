\documentclass{article}
\usepackage[utf8]{inputenc}

\title{reverse}
\author{wright.aprilm }
\date{October 2019}

\begin{document}

\maketitle
\section{Introduction}

\begin{itemize}
    \item Why do people need phylogenies?
    \item What unique stuff does RevBayes bring to phylogenetic research? 
\end{itemize}

\section{The Context of Systematics Education}

\begin{itemize}
    \item i.e., why do we even need a GUI
    \item Some biology disciplines routinely use server and command-line interfaces
    \item This is less common in others
    \item In courses where the content, not the code, is the focus, GUIs more common.
    \item Not uncommon to have courses with multiple skill levels among learners
    \item RevBayes relatively high cognitive load
    \item Python is more common in large data \& bioinformatics
    \item R more common in paleo \& comparative biology
\end{itemize}

\section{RevStudio}

How does this work?  
How can this tool be used in making websites, manuals, etc == interoperability

\section{RevNotebook}

How does this work? 

\section{A Comparison of the three interfaces}

When I use which?


\end{document}
